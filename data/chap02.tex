% !TeX root = ../main.tex

\chapter{持久性内存文件系统可用性技术简介}
\label{cha:tech}

% 使用的原理和技术的介绍

本章介绍本文涉及的的存储、网络和分布式文件系统相关技术。

\section{持久性内存技术}
\label{sec:ch2_nvm}

在本文中,持久性内存(NVM)特指在断电时不会丢失数据的随机访问存储器。NVM 的一个重要特性是支持字节粒度的随机访问,这使它与必须以 4KB 或其他大粒度读写的闪存(Flash Memory)等技术区别开来。

字节粒度访问特性使 NVM 提供了极大的灵活性,但这也造成它的生产技术难度较大,且成本昂贵。常见的 NVM 技术包括铁电存储器(FeRAM)、磁阻存储器(MRAM)、相变存储器(PCM)、铁电栅场效应晶体管存储器(FeFET Memory)等。例如,美国 Ramtron、Texas Instruments 等公司已经量产用于工业设备的 FeRAM 存储设备,美国 Everspin 公司也已量产容量为 4Mb 的 MRAM 存储器。对于分布式文件系统领域来说,这些产品的存储容量都过于低下,缺少实用价值。

虽然没有适合的 NVM 设备,但近年来 NVM 技术仍然发展迅速,文件系统领域的许多最新成果使用了 NVM 作为存储介质(之一)。这是因为 Linux 内核已经提供了针对 NVM 的支持,并且提供了将 DRAM 内存空间映射为 NVM 设备的接口,大部分工作利用该接口通过大容量 DRAM 模拟 NVM。由于 NVM 的访问特性与 DRAM 并不完全相同,内存空间映射并不能解决这个问题,因此也有工作\cite{gogtewear2019}利用远程节点内存模拟 NVM,通过网络延迟模拟 NVM 的读写延迟。  

NVM 技术真正得到商用是在 2019 年,由 Intel 公司宣布推出使用 PCM 技术的 Optane DC 持久性内存产品;其容量可达 256GB,真正具备了取代传统存储设备的潜力。真实产品的出现也使得对 NVM 读写特性的详细测试成为可能。

\section{远程直接内存访问技术}
\label{sec:ch2_rdma}

远程直接内存访问(RDMA)指的是一种通过支持 RDMA 的网卡(即 RDMA NIC),直接读写远端节点内存的技术。与直接内存访问(DMA)技术类似,RDMA 在传输数据时无需操作系统参与,减少了上下文切换和数据复制的 CPU 开销,相较于传统 TCP/IP 协议栈而言,极大地降低了网络之间数据传输的延迟。

RDMA 技术支持许多十分有用的特性:

\begin{enumerate}[(1)]

    \item 与 TCP 协议类似,RDMA 也支持可靠连接(Reliable Connection);两个节点之间的RDMA 连接建立后,如果其中一方意外断开,另一方能立即收到连接断开的消息。这使分布式系统能方便地检测到其中某个节点的故障;
    \item 与套接字类似,RDMA 支持发送(send)/接收(recv)原语,能够以通知远端节点的方式传送数据,即双边操作;RDMA 也支持读(read)/写(write)原语,能够在不通知远端节点的方式传送数据,即单边操作。由于双边操作适合实现远程过程调用,单边操作适合进行单纯的数据传输,文件系统可以使用合适的原语来优化其性能;
    \item RDMA 同时支持一种轻量级的通知远端节点的写原语(write-with-imm),这使得文件系统中的节点能高效地监控到达其上的写入操作并记录日志,从而方便地实现节点的错误恢复。

\end{enumerate}

\section{数据冗余技术}
\label{sec:ch2_avail}

可用性的定义是系统在特定时间段内能正常提供服务的能力。在分布式系统中,组成系统的各个存储节点彼此在物理上独立,这使得某个节点发生故障时其他节点能不受波及,但也使得整个系统的故障率极大提升。例如,假如某份数据在整个系统中仅存有一个副本,则当存放该副本的节点发生故障时,数据就不再可用。为了解决这一问题,文件系统势必要付出额外的空间存储冗余数据;由于冗余数据的存在,系统也因而必须付出额外的 CPU 和网络开销。

通过存储冗余数据为系统提供可用性的技术即称为数据冗余技术。不同的数据冗余技术在 CPU、网络通信和存储空间开销之间作出不同的权衡,因此也适合不同的场景。本课题中涉及的数据冗余技术为以下 2 种:数据备份,以及纠删码。

\subsection{数据备份}
\label{subsec:ch2_avail_repli}

数据备份(Replication)是一种最为常见和简单的数据冗余技术。它的基本思想是,对于存入文件系统的一份数据,系统实际上将它存储为 $r$ 个完全相同的副本,分别位于物理上分离的 $r$ 台存储设备上。只要 $r$ 份数据没有全部损坏,系统总能从中选择一份完好的数据副本用于提供服务。因此,系统最多能容忍 $p = r - 1$ 个节点同时故障。实际使用时常常取 $r = 3$,对应的策略称为三副本策略。

数据备份的优点是简单、直接;在写数据时,只需要将数据原封不动写入到若干台存储设备中,读数据时则从其中任意选择一个,读出原始数据即可。因此,它除了在写入时需要进行额外的网络访问,几乎不会产生其他的开销。

数据备份的缺点则是会占用大量的存储空间。例如,使用三副本策略时,全部存储空间只有三分之一可能被有效利用,其余部分必须用来存储数据的备份。在大容量、价格低廉的 HDD 或 SSD 上,备份策略不失为一种经济而行之有效的手段;然而在 NVM 上,它则会大量占用 NVM 本就十分昂贵的存储空间,进一步加大数据的存储成本。表 ~\ref{tab:dev_price} 给出了典型的 HDD、SSD 和 NVM 目前的价格,NVM 单位容量的价格相比 SSD 高出 65 倍。如果在 NVM 上随意应用数据备份策略,则必然导致昂贵的存储开销。

\begin{table}[htb]
    \centering
    \caption[存储介质单位价格对比]{存储介质单位价格对比}
    \label{tab:dev_price}
      \begin{tabular}{clc}
        \toprule[1.5pt]
        {\heiti 存储介质} & {\heiti 设备型号} & {\heiti 每 GB 价格(美元)} \\\midrule[1pt]
        HDD & Seagate Ironwolf 1TB NAS Raid & 0.07 \\
        SSD & Intel\textregistered ~SSD 760p 1024GB & 0.16 \\
        NVM & Intel\textregistered ~Optane\textsuperscript{\texttrademark} DC Persistent Memory 256GB & 10.42 \\
        \bottomrule[1.5pt]
      \end{tabular}
\end{table}

本课题实现了一个简单的三副本策略,作为性能测试结果的基线对照。

\subsection{纠删码}
\label{subsec:ch2_avail_ec}

纠删码(Erasure Code)是另外一种常见的数据冗余技术,广泛使用于网络通信数据编码等领域。它的基本思想是将 $k$ 份原始数据通过一定的方式进行编码,得到 $p$ 份冗余数据,合计 $n = k + p$ 份数据,分别位于物理上分离的 $n$ 台存储设备上。当其中至多 $p$ 份任意数据出错时,均可以通过算法重构出原始的 $k$ 份数据。因此,系统能

Reed-Solomon 编码是最典型的纠删码策略之一,其对应的编码策略通常记为 $RS(k, p)$。一般来说,它构建一个 Vandermonde 矩阵 $V$ 用于编码:
\begin{equation}
\label{equ:vandermonde}
    V = \left(
\begin{array}{ccccc}
1 & a_1 & a_1^2 & \cdots & a_1^{k-1} \\
1 & a_2 & a_2^2 & \cdots & a_2^{k-1} \\
\vdots & \vdots & \vdots & \ddots & \vdots \\
1 & a_p & a_p^2 & \cdots & a_p^{k-1}
\end{array}
    \right)
\end{equation}

其中 $a_1, a_2, \cdots, a_p$ 互不相同且均不为 0。容易看出,数据备份可以看成是纠删码的一个特例:当 $a_1 = 1, k = 1$ 时,纠删码策略 $RS(1, p)$ 退化为 $r = p + 1$ 的数据备份策略。

实际操作时,可以用其他列满秩矩阵代替 Vandermonde 矩阵来计算 $V$,有时能带来性能上的提升。

纠删码的优点是节省存储空间。为了衡量它节省空间的比例,我们定义存储效率等于数据的实际大小和它耗费的存储空间的比值。存储效率越高,节省的空间就越多。例如,三副本策略能容忍至多 2 个节点同时故障;其代价是将一份数据重复存储在 3 个位置上,存储效率为 33.3\%。然而,任意的 $RS(k, 2)$ 同样能容忍 2 个节点同时故障;其代价是每 $k$ 份原始数据都存储额外的 2 份校验数据,存储效率等于 $\frac{k}{k + 2}\times 100\%$。显然,该值随 $k$ 的增大而增大;只要 $k > 1$,换言之,只要纠删码策略不退化为数据备份策略,它就能带来更高的存储效率。实践中,$k$ 能够取到 10\cite{infinicache2020},对应的存储效率为 83.3\%,为三副本策略的 2.5 倍。

纠删码的缺点则是需要引入额外的计算量。例如,对于 Reed-Solomon 码来说,它的编码和解码过程都涉及到矩阵运算,而数据备份策略只需进行网络传输,几乎完全没有计算开销。对此,已有的一些观点\cite{octopus2017,orion2019}认为,相比于传统的块存储设备和基于 TCP/IP 协议栈的网络,NVM 和 RDMA 的引入已经极大地降低了存储和网络通信的延迟;为了继续提升性能,有必要减少耗费在 CPU 计算上的时间。根据这种观点,如果使用反而增加 CPU 计算开销的纠删码,将会是一个不经济的选择。

本文的目标在于证明,上述观点并不完全正确。使用纠删码虽然会引入额外的 CPU 开销,但其性能表现仍然能与传统的数据备份策略媲美。
