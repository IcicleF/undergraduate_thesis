% !TeX root = ../main.tex

\chapter{技术选型简介}
\label{cha:tech}

% 使用的原理和技术的介绍

本章介绍本文涉及的的存储、网络和分布式文件系统相关技术,同时详细介绍本课题使用的技术选型。

\section{持久性内存技术}
\label{ch2_nvm}

持久性内存(NVM)指的是在断电时不会丢失数据的随机访问存储器。NVM 的一个重要特性是支持字节粒度的随机访问,这使它与必须以 4KB 或其他大粒度读写的闪存(Flash Memory)等技术区别开来。

常见的 NVM 技术包括铁电存储器(FeRAM)、磁阻存储器(MRAM)、相变存储器(PCM)、铁电栅场效应晶体管存储器(FeFET Memory)等。NVM 技术在近年来发展迅速。例如,美国 Ramtron、Texas Instruments 等公司已经量产 FeRAM 存储设备,美国 Everspin 公司也已量产容量为 4Mb 的 MRAM 存储器。由于它们的存储容量都较小,因此 NVM 技术真正得到商用是在 2019 年,由 Intel 公司宣布推出使用 PCM 技术的 Optane DC 持久性内存产品;其容量可达 256GB,真正具备了取代传统存储设备的潜力。

本课题目前使用 DRAM 模拟 NVM 设备。本课题选取的实验环境操作系统为 CentOS 7,其支持将一定容量的 DRAM 模拟为 NVM 空间,并提供与真实 NVM 相同的设备接口。

\section{远程直接内存访问技术}
\label{ch2_rdma}

远程直接内存访问(RDMA)指的是一种通过支持 RDMA 的网卡(即 RDMA NIC),直接读写远端节点内存的技术。与直接内存访问(DMA)技术类似,RDMA 在传输数据时无需操作系统参与,减少了上下文切换和数据复制的 CPU 开销,相较于传统 TCP/IP 协议栈而言,极大地降低了网络之间数据传输的延迟。

RDMA 技术支持许多十分有用的特性:

\begin{enumerate}[(1)]

    \item 与 TCP 协议类似,RDMA 也支持可靠连接(Reliable Connection);两个节点之间的RDMA 连接建立后,如果其中一方意外断开,另一方能立即收到连接断开的消息。这使分布式系统能方便地检测到其中某个节点的故障;
    \item 与套接字类似,RDMA 支持发送(send)/接收(recv)原语,能够以通知远端节点的方式传送数据,即双边操作;RDMA 也支持读(read)/写(write)原语,能够在不通知远端节点的方式传送数据,即单边操作。由于双边操作适合实现远程过程调用,单边操作适合进行单纯的数据传输,文件系统可以使用合适的原语来优化其性能;
    \item RDMA 同时支持一种轻量级的通知远端节点的写原语(write-with-imm),这使得文件系统中的节点能高效地监控到达其上的写入操作并记录日志,从而方便地实现节点的错误恢复。

\end{enumerate}

\section{数据冗余技术}
\label{ch2_avail}

可用性的定义是系统在特定时间段内能正常提供服务的能力。在分布式系统中,组成系统的各个存储节点彼此在物理上独立,这使得某个节点发生故障时其他节点能不受波及,但也使得整个系统的故障率极大提升。例如,假如某份数据在整个系统中仅存有一个副本,则当存放该副本的节点发生故障时,数据就不再可用。在不应用数据压缩算法的前提下,解决这一问题势必要付出额外的存储空间存储冗余数据,从而也就必须付出额外的 CPU 和网络开销。

在 CPU、网络和存储空间开销之间作出权衡,通过存储冗余数据为系统提供可用性的技术即称为数据冗余技术。

\subsection{数据备份}
\label{ch2_avail_repli}

数据备份(Replication)是一种最为常见和简单的数据冗余技术。它的基本思想是,对于存入文件系统的一份数据,系统实际上将它存储为 $r$ 个完全相同的副本,分别位于物理上分离的 $r$ 台存储设备上。只要 $r$ 份数据没有全部损坏,系统总能从中选择一份完好的数据副本用于提供服务。实际使用时通常取 $r = 3$,对应的策略称为三副本策略。

数据备份的优点是简单、直接;例如,写数据时只需要将数据原封不动写入到若干个设备中,读数据时则从中任意选择一个,读出原始数据即可。

它的缺点则是存储效率低下。例如,使用三副本策略时,全部存储空间只有三分之一被有效利用,其余都用来存储数据的备份。在大容量、价格低廉的 HDD 或 SSD 上,备份策略不失为一种行之有效的手段;然而在 NVM 上,它则会大量占用 NVM 本就十分昂贵的存储空间,进一步加大数据的存储成本。

本课题实现了数据备份策略,作为实验结果的基线对照。

\subsection{纠删码}
\label{ch2_avail_ec}

纠删码(Erasure Code)是另外一种常见的数据冗余技术,广泛使用于网络通信数据编码等领域。它的基本思想是将 $k$ 份原始数据通过一定的方式进行编码,得到 $p$ 份冗余数据,合计 $k + p$ 份数据,分别位于物理上分离的 $k + p$ 台存储设备上。当其中至多 $p$ 份任意数据出错时,均可以通过算法重构出原始的 $k$ 份数据。典型的纠删码策略如 Reed-Solomon 编码,对应的编码策略通常记为 $RS(k, p)$。容易看出,上述数据备份策略实际上等价于 $RS(1, r - 1)$。

纠删码的优点是节省存储空间、存储效率高。例如,三副本策略能容忍至多 2 个节点同时故障,存储效率为三分之一。然而,任意的 $RS(k, 2)$ 同样能容忍 2 个节点同时故障,它的存储效率等于 $\frac{k}{k + 2}$。实践中 $k$ 能够取到 10\cite{wang2020},对应的存储效率约为 83.3\%,为三副本策略的 2.5 倍。

纠删码的缺点则是需要引入额外的计算量。例如,对于 Reed-Solomon 编码来说,它的编码和解码过程都涉及到矩阵运算。编码 $k$ 份长度为 $L$ 的原始数据的计算复杂度为 $O((k+p)kL)$;解码则首先需要花费至少 $O(k^2)$ 的时间计算解码矩阵,再花费 $O(pkL)$ 的时间解码原始数据。虽然实践中 $k$、$p$ 并不大,但它表明纠删码策略会引入正比于数据量的 CPU 开销,而数据备份策略几乎没有这部分开销。

本课题实现了纠删码策略,并且将会通过实验证明:纠删码编解码虽然带来了额外的 CPU 计算量,但它对系统的影响是可忽略的。
