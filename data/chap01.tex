% !TeX root = ../main.tex

\chapter{引言}
\label{cha:intro}

% 本课题问题的提出、意义及实用价值;已有研究情况的概述;本课题所要解决的问题等。

\section{研究背景}
\label{sec:ch1_background}

文件是用于数据处理的最基本的抽象之一。日益增长的互联网规模正带来日益增长的文件存储需求;单机文件系统早已无法应对如此巨大的数据量,分布式文件系统(Distributed File System,DFS)因而应运而生。相比于单机系统,DFS 具有高性能、高并发度和良好的可扩展性,已经成为近年来的热点研究方向。

DFS 通常包含三个最主要的部分:存储、计算和网络通信,系统的性能由这三者共同决定。由于其访问特性不同,平衡其性能从而优化整个 DFS 成为了近年来该领域研究的一个主流方向。另外,组成 DFS 的存储节点在物理上彼此分离;在提供单机无法比拟的数据处理能力的同时,它们也可能各自独立地发生故障,从而大大提升整个系统发生故障的概率。因此,DFS 通常必须引入额外的存储、计算和网络通信,从而提供可用性保障。即,系统在不超过一定数量的节点发生故障时,应当继续保持正常工作。如何在系统的额外开销和其可用性等级之间权衡,是 DFS 设计中一个必须考虑的问题。

传统 DFS 通常使用机械式磁盘(Hard Disk Drive,HDD)或固态硬盘(Solid State Drive,SSD)作为其持久性存储介质。其优点是容量较大,如 Seagate 公司已经生产出容量达 16TB 的商用 HDD,SanDisk 公司也已制造出容量达 8TB 的 SSD 原型。其缺点是访问延迟也较大,如 HDD 的随机访问由于寻道、磁盘旋转等延迟的存在,通常需要花费数毫秒;SSD 通常必须以 4KB 为单位读写数据,平均访问延迟通常达到数百微秒。这种访问特性使得存储部分成为传统 DFS 的瓶颈。典型的此类 DFS 如 Ceph\cite{ceph2006}、HDFS\cite{hadoop2010} 和 Gluster\cite{gluster2013}。这些 DFS 通常针对系统中延迟最高的存储介质读写操作进行优化,同时大量占用并非瓶颈的 CPU 资源(例如,Ceph 的 compaction 操作可产生持续 20 秒以上的 CPU 满载\cite{assise2019})。为了保证可用性,上述 DFS 通常存储数据的若干个备份,以防单节点故障导致数据无法访问。

近年来,持久性内存(Non-volatile Memory,NVM)技术的迅速发展打破了这种设计范式。与以 HDD、SSD 为代表的传统持久性存储介质相比,NVM 在断电不丢失数据的同时,支持字节粒度访问,并且读写延迟低 3 $\sim$ 4 个数量级。例如,Intel 公司 Optane\textsuperscript{\texttrademark} SSD 产品的 4KB 随机读写延迟约为 $\SI{214}{\us}$\cite{optanessd},而其 Optane\textsuperscript{\texttrademark} DC NVM 产品的 4KB 随机读延迟约为 0.06 $\sim \SI{0.3}{\us}$\cite{yangnvm2020}。这种情况下,传统 DFS 对存储介质读写的优化无法起到效果,CPU 和网络通信开销反而成为系统的瓶颈。因此,面向低速存储介质的传统 DFS 在 NVM 面前不再适用,发展适合 NVM 特性的新型高可用 DFS 势在必行。

基于 NVM 的分布式文件系统需要面对的一个核心问题是其存储成本。由于 HDD、SSD 技术都早已成熟,单位容量价格十分低廉,传统 DFS 可以存储数据的多个副本,而不必担心因此产生过多的存储成本。然而,NVM 单位容量价格则较为昂贵;如果仍然采用存储多个数据副本的可用性策略,将会数倍提高本已十分高昂的存储成本。如何在保证 DFS 高效率的同时尽可能减少存储成本,仍然是一个开放问题,既往研究几乎没有对其进行讨论。

纠删码是一种较为新型的数据容错策略,在电子通信领域已经得到广泛应用,因其只需额外传输很少的数据,就能提供较高的容错能力。在节省存储成本方面,纠删码同样有着巨大的潜力。文件系统研究领域中,已有许多工作使用了纠删码,但其在基于 NVM 的分布式文件系统中表现如何尚未有定论。因此,本课题的目标是在分布式 NVM 环境下优化、实现纠删码策略,以此为文件系统中的数据提供可用性保障,并测试其性能表现。

\section{研究现状}
\label{sec:ch1_relworks}

本小节从基于 NVM 的文件系统和基于纠删码的系统可用性策略两方面入手,介绍学术界和工业界的研究现况。

\subsection{基于 NVM 的文件系统}
\label{subsec:ch1_nvm_relworks}

NVM 技术在近 2 $\sim$ 3 年发展迅速。初期一些工作提出使用 NVM 代替 HDD、SSD 等低速存储介质,从而提高单机文件系统的性能。此类成果如加州大学圣迭戈分校提出的 NOVA\cite{nova2016}、NOVA-Fortis\cite{novafortis2017},以及德克萨斯大学奥斯汀分校提出的 Strata\cite{strata2017}。NOVA 利用了 NVM 的字节粒度访问的特性,将文件系统日志存储在 NVM 中,减少日志持久化带来的开销;NOVA-Fortis 在此基础上提供了数据完整性的保障。Strata 则创新地提出了一个基于 NVM、SSD 和 HDD 的混合介质存储系统,以期综合利用三者各自的性能优势。

另外,也已有多项研究着眼于使用 NVM 作为 DFS 中的持久性存储介质。由于存储介质的读写延迟已经极大降低,传统的基于 TCP/IP 协议栈的网络通信成为了新的性能瓶颈之一。新出现的远程内存直接访问(Remote Direct Memory Access,RDMA)技术支持在用户态直接访问远端节点的内存,减少了数据复制和上下文切换开销,极大提升了网络带宽,基本解决了这一问题。综合利用 NVM 和 RDMA 技术的现有工作如清华大学提出的 Octopus\cite{octopus2017}、加州大学圣迭戈分校提出的 Orion\cite{orion2019} 和华盛顿大学提出的 Assise\cite{assise2019}。这些工作从不同方面优化 DFS 的性能。例如,Octopus 改进了分布式事务协议以降低网络请求开销;Orion 借鉴了 NOVA,通过高效、中心化的元数据服务降低维持一致性的开销;Assise 构建了完善的多级缓存和备份机制,以此实现快速的错误恢复。

然而,在系统可用性方面,上述研究均使用了备份策略,均忽视了高成本 NVM 带来的节省存储空间的需求。尚未有已发表的工作研究如何以较低的 NVM 空间开销保证系统的高可用性。

\subsection{基于纠删码的系统可用性策略}
\label{subsec:ch1_ec_relworks}

纠删码早在 1950 年代就出现在通信领域中。其代表是 Hamming 码,可以识别和纠正通信中的任意单比特错误。在存储领域,纠删码同样已经广泛应用在学术界前沿研究以及工程实践中。

学术研究方面,上海交通大学提出的 Cocytus\cite{cocytus2016} 在一个单机内存键值存储引擎中应用了纠删码策略,以应对工作进程的意外终止;乔治·梅森大学提出的 InfiniCache\cite{infinicache2020} 构建了一个大规模分布式数据缓存系统,同样应用了纠删码以防止数据丢失、提升效率及降低成本;密歇根大学提出的 Hydra\cite{hydra2019} 则实现了一个使用纠删码策略的分布式内存分配器,用以降低访问远程内存的延迟。以上研究表明了在高性能系统中使用纠删码的可行性和有效性。

工程实践方面,微软公司早在 2012 年就在其 Windows Azure 云存储系统中部署了一个低错误恢复开销的纠删码策略\cite{azure2012},用于应对 Azure 存储服务器频繁的下线更新,使系统始终保持可用。Facebook 公司使用名为 f4 的基于纠删码的存储系统\cite{facebook2014},用于以较少的空间存储 Facebook 上大量的图片和流媒体文件。以上实例表明了纠删码用于文件系统时带来的明显优势。

然而,上述系统或者不是标准的 DFS,不能代表纠删码在文件系统中的性能表现;或者面向传统存储介质和 TCP/IP 协议栈设计,不能适应基于 NVM+RDMA 的存储环境。尚未有已发表的工作研究如何在高性能存储和网络环境中应用纠删码策略。

\section{本课题解决的问题}
\label{sec:ch1_thiswork}

既往研究已经揭示了纠删码作为分布式存储系统的可用性保障的潜力。但是,尚未有研究在 NVM+RDMA 环境下构建一个使用纠删码的 DFS,通过实证的方式检验纠删码的可行性和有效性。本课题首先作出了填补这一空白的尝试。具体而言,本课题解决的问题是:证明在 NVM 作为存储介质、RDMA 提供高速高带宽网络的环境下,纠删码是一种合适的可用性方案;其能在节省存储空间的同时,又提供较好的性能表现。

为了实现上述研究目标,本课题采取以下的研究方案:首先,设计并实现一个基于 NVM 和 RDMA 的新型分布式文件系统。该系统采用模块化的设计方式,针对 NVM+RDMA 环境对现有的纠删码策略进行优化、移植。完成系统的设计后,从理论上证明,相较于传统数据备份策略,使用纠删码能带来更低的存储开销。完成系统的实现后,对该系统开展综合的测试,与传统数据备份策略对比,证明纠删码能提供与之相当的可用性保障,同时表现出与之相当或更优的性能。
