% !TeX root = ../main.tex

\chapter{引言}
\label{cha:intro}

% 本课题问题的提出、意义及实用价值;已有研究情况的概述;本课题所要解决的问题等。

\section{研究背景}
\label{sec:ch1_background}

文件是用于数据处理的最基本的抽象之一。日益增长的互联网规模正带来日益增长的文件存储需求;单机文件系统早已无法应对如此巨大的数据量,分布式文件系统(Distributed File System,DFS)应运而生。相比于单机系统,DFS 具有高性能、高并发度和良好的可扩展性,已经成为近年来的热点研究方向。

DFS 通常包含三个最主要的部分:存储、计算和网络通信,系统的性能由这三者共同决定。由于三者各自的性能互不相同,平衡三者的性能从而优化整个系统成为了近年来分布式文件系统方向研究的主流方向。另外,组成 DFS的单机彼此之间在物理上分离。在共同提供单机所无法比拟的数据处理能力的同时,它们也可能各自独立地发生故障,从而使得整个系统发生故障的概率大大提高。因此,DFS 通常必须引入额外的存储、计算和网络通信,从而提供可用性保障,即是说,系统在某些节点故障时应该仍然能正常工作。如何在三者性能损失和系统可用性之间权衡,是 DFS 设计中另一个必须考虑的问题。

传统 DFS 通常使用机械式磁盘(Hard Disk Drive,HDD)或固态硬盘(Solid State Drive,SSD)作为其持久性存储介质。它们的优点是容量较大:如 Seagate 公司已经生产出容量达 16TB 的商用 HDD,SanDisk 公司也已制造出容量达 8TB 的 SSD 原型;缺点则是访问延迟也较大:HDD 的随机访问由于寻道、磁盘旋转等延迟的存在通常需要花费数个毫秒,SSD 的平均访问延迟也通常达到数百微秒。这种访问特性使得存储部分成为整个 DFS 的瓶颈。基于传统存储介质的 DFS 的典型例子如 Ceph\cite{ceph2006}、HDFS\cite{hadoop2010} 和 Gluster\cite{gluster2013}。它们通常针对系统中性能最低的存储介质读写进行优化,同时占用 CPU 进行大量运算(例如,Ceph 定期执行的 compaction 操作会带来巨大 CPU 负载)。为了保证可用性,这些文件系统通常会存储数据的若干个备份,防止单节点故障导致数据无法访问。

近年来,持久性内存(Non-volatile Memory,NVM)技术的迅速发展打破了这种设计范式。与传统的持久性存储介质相比,NVM 同样具有断电不丢失数据的优点;同时,它支持字节粒度访问,读写延迟低三到四个数量级。例如,Intel 公司 Optane\textsuperscript{\texttrademark} SSD 产品的 4KB 随机读写延迟约为 $\SI{214}{\us}$\cite{optanessd},而其 Optane\textsuperscript{\texttrademark} DC NVM 产品的 4KB 随机读延迟约为 $\SI{0.3}{\us}$,随机写速度更是低至 0.06 $\sim \SI{0.09}{\us}$\cite{yangnvm2020}。在这种情况下,传统 DFS 对存储介质读写的优化就显得无足轻重;而同时,CPU 的计算资源和网络带宽反而成为了新的瓶颈。因此,基于大容量、低速存储介质的传统 DFS 在 NVM 的这些新特性面前不再适用。发展适合 NVM 特性的新型高可用 DFS势在必行。

基于 NVM 的分布式文件系统的一个核心问题是其存储成本。由于 HDD、SSD 技术都早已成熟,单位容量价格十分低廉,传统 DFS 可以存储数据的多个副本,而不必担心因此产生过多的额外花销。NVM 单位容量的价格则较为昂贵,如果直接应用既有的可用性策略,则势必产生巨大的存储成本。如何在保证 DFS 高效率的同时尽可能减少存储开销,仍然是一个开放问题,既往研究几乎没有涉及。

纠删码是一种较为新型的数据容错策略,它在电子通信领域已经得到广泛应用:只需额外传输很少的额外数据,就能提供较高的容错能力。因此,在节省存储开销方面,纠删码有着巨大的潜力。在文件系统领域,也已有许多工作使用了纠删码,但它在基于 NVM 的分布式系统中表现如何尚未有定论。因此,本课题的目标是在 NVM+RDMA 环境下实现纠删码,并以此为文件系统中的数据提供可用性保障,最后测试其性能表现。

\section{研究现状}
\label{sec:ch1_relworks}

本小节从基于 NVM 的文件系统和纠删码两方面介绍学术界目前的研究情况。

\subsection{基于 NVM 的文件系统}
\label{subsec:ch1_nvm_relworks}

NVM 技术在近两三年发展迅速。早期的一些工作提出使用 NVM 代替 HDD、SSD 等低速存储介质,从而提高单机文件系统的性能。已有的成果如加州大学圣迭戈分校提出的 NOVA\cite{nova2016} 和它的改进 NOVA-Fortis\cite{novafortis2017}。NOVA 利用了 NVM 的字节粒度访问的特性,将文件系统日志存储在 NVM 中,减少了日志持久化带来的开销。NOVA-Fortis 在此基础上还提供了数据完整性的保障。

另外,也已有多项研究开始使用 NVM 作为 DFS 中的持久性存储介质。由于数据存储的延迟已经极大地降低,传统的基于 TCP/IP 网络协议栈的网络通信成为了新的性能瓶颈。新兴的远程内存直接访问(Remote Direct Memory Access,RDMA)技术支持用户态直接访问远端节点的内存,减少了大量的数据复制,很大程度上缓解了这一问题。综合利用 NVM 和 RDMA 技术的已有成果如清华大学提出的 Octopus\cite{octopus2017}、加州大学圣迭戈分校提出的 Orion\cite{orion2019} 和华盛顿大学提出的 Assise\cite{assise2019}。它们从不同方面入手优化文件系统的性能。例如,Octopus 改进了分布式事务协议来降低网络请求的开销;Orion 借鉴了 NOVA,通过一个高效的元数据服务器降低维持一致性的开销;Assise 则实现了完善的缓存机制,通过多级备份实现快速的错误恢复。

然而,上述这些研究使用的仍是备份策略,忽视了当前高成本 NVM 带来的节省存储空间的需求。尚未有已发表的工作研究如何以较低的 NVM 空间开销保证系统的高可用性。

\subsection{基于纠删码的文件系统可用性策略}
\label{subsec:ch1_ec_relworks}

纠删码早在 1950 年代就出现在通信领域,其代表是 Hamming 码,可以识别和纠正通信中的任意单比特错误。在存储领域,纠删码已经被应用于前沿研究和工程实际中。

前沿研究方面,上海交通大学提出的 Cocytus\cite{cocytus2016} 在一个内存键值存储引擎中应用了纠删码策略;乔治·梅森大学提出的 InfiniCache\cite{infinicache2020} 构建了一个大规模分布式数据缓存系统,同样应用了纠删码,以提供数据可用性保障、提升效率和降低成本;密歇根大学提出的 Hydra\cite{hydra2019} 则构建了一个使用纠删码策略的分布式内存分配器。这些研究表明了在高性能系统中使用纠删码的可行性和有效性。

工程实际方面,微软公司早在 2012 年就在其 Windows Azure 云存储系统中部署了一个低开销的纠删码策略\cite{azure2012},用于应对 Azure 存储服务器频繁的下线更新,使系统始终保持可用。Facebook 公司也使用了一个称为 f4 的纠删码存储系统\cite{facebook2014},用于以较低的空间开销存储 Facebook 上大量的图片和流媒体文件。这些纠删码的应用实例表明了纠删码用于文件系统时带来的明显优势。

然而,这些系统或者不是标准的分布式文件系统,不能代表纠删码在文件系统中的性能表现;或者基于传统块设备设计,并且已经有一定的历史,不能适应基于 NVM+RDMA 的最新存储环境。尚未有已发表的工作研究如何在高性能存储和网络环境下应用纠删码策略。

\section{本课题解决的问题}
\label{sec:ch1_thiswork}

既往研究已经揭示了纠删码作为分布式存储系统的可用性保障的潜力。但是,尚未有研究在 NVM+RDMA 环境下构建一个使用纠删码的 DFS,通过实证的方式检验纠删码的可行性和有效性。本课题首先作出了填补这一空白的尝试。具体而言,本课题解决的问题是:证明在 NVM 作为存储介质、RDMA 提供高速高带宽网络的环境下,纠删码是一种合适的可用性方案,其能在节省存储空间的同时,又保证较好的性能。

为了实现上述研究目标,本课题采取了以下的研究方案:首先,设计并实现一个基于 NVM 和 RDMA 的新型分布式文件系统。该系统采用模块化的设计方式,针对 NVM+RDMA 环境对现有的纠删码策略进行优化、移植。完成系统的设计后,从理论上证明,相较于传统数据备份策略,使用纠删码能带来更低的存储开销。完成系统的实现后,对该系统开展综合的测试,与传统数据备份策略对比,证明纠删码能提供与之相当的可用性保障,同时也能提供与之相当或更优的性能。
