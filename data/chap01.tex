% !TeX root = ../main.tex

\chapter{引言}
\label{cha:intro}

% 本课题问题的提出、意义及实用价值;已有研究情况的概述;本课题所要解决的问题等。

\section{研究背景}
\label{sec:ch1_background}

文件是用于数据处理的最基本的抽象之一。日益增长的互联网规模正带来日益增长的文件存储需求;单机文件系统早已无法应对如此巨大的数据量,分布式文件系统(Distributed File System,DFS)应运而生。相比于单机系统,DFS 具有高性能、高并发度和良好的可扩展性,已经成为近年来的热点研究方向。

分布式文件系统通常包含三个最主要的部分:存储、计算和网络通信,系统的性能由这三者共同决定。由于它们各自的性能互不相同,平衡三者的性能从而优化整个系统成为了近年来分布式文件系统方向研究的主流方向。另外,组成分布式系统的单机彼此之间在物理上分离。在共同提供单机所无法比拟的数据处理能力的同时,它们也可能各自独立地发生故障,从而使得整个系统发生故障的概率大大提高。因此,分布式文件系统通常必须引入额外的存储、计算和网络通信,从而提供可用性保障,即是说,系统在某些节点故障时应该仍然能正常工作。如何在三者性能损失和系统可用性之间权衡,是 DFS 设计中另一个必须考虑的问题。

传统分布式文件系统通常使用机械式磁盘(Hard Disk Drive,HDD)或固态硬盘(Solid State Drive,SSD)作为其持久性存储介质。它们的优点是容量较大:如 Seagate 公司已经生产出容量达 16TB 的商用 HDD,SanDisk 公司也已制造出容量达 8TB 的 SSD 原型;缺点则是访问延迟也较大:HDD 的随机访问由于寻道、磁盘旋转等延迟的存在通常需要花费数个毫秒,SSD 的平均访问延迟也通常达到数百微秒。这种访问特性使得存储部分成为整个分布式文件系统的瓶颈,也使传统 DFS 集中于优化其对存储介质的访问。这类 DFS 的典型例子如 Ceph\cite{ceph2006}、HDFS\cite{hadoop2010} 和 Gluster\cite{gluster2013}。

近年来,持久性内存(Non-volatile Memory,NVM)技术的迅速发展打破了这种设计范式。与传统的持久性存储介质相比,NVM 同样具有断电不丢失数据的优点;同时,它支持字节粒度访问,读写延迟也相对较低。对比之下,HDD 必须以 512 字节的扇区为单位读写数据,SSD 则通常必须以 4KB 的数据页为单位进行读写。

因此,基于大容量、低速存储介质的传统 DFS 在 NVM 的这些新特性面前不再适用。同时,NVM 也为高效、低成本地实现高可用性提供了新的机遇。发展适合 NVM 特性的新型高可用分布式文件系统势在必行。

\section{NVM 对 DFS 提出的挑战}
\label{sec:ch1_challenge}

NVM 的读写延迟相比 SSD 低三到四个数量级。例如,Intel 公司 Optane SSD 产品的 4KB 随机读写延迟约为 214 us\cite{optanessd},而其 Optane DC NVM 产品的 4KB 随机读延迟约为 0.3 us,随机写速度更是低至 0.06 $\sim$ 0.09 us\cite{yang2020}。在这种情况下,传统分布式文件系统对存储介质读写的优化就显得无足轻重;同时,计算资源(即 CPU)和网络带宽反而成为了新的瓶颈。

NVM 的主要缺点就是单位容量的成本远高于 HDD 和 SSD;例如,Intel 公司的 760p SSD 产品每 GB 价格约为 0.16 美元,而其 Optane DC NVM 产品每 GB 价格高达 10.42 美元,是前者的 65 倍。传统 DFS 通常不会特意节省廉价的硬盘存储空间,而基于 NVM 的 DFS 则不得不考虑这一问题。

\section{研究现状}
\label{sec:ch1_relworks}

NVM 技术在近两三年发展迅速。早期的一些工作提出使用 NVM 代替 HDD、SSD 等低速存储介质,从而提高单机文件系统的性能。已有的成果如 NOVA\cite{nova2016} 和它的改进 NOVA-Fortis\cite{novafortis2017}。NOVA 利用了 NVM 的字节粒度访问的特性,将文件系统日志存储在 NVM 中,减少了日志持久化带来的开销。NOVA-Fortis 在此基础上还提供了数据完整性的保障。

另外,也已有多项研究开始使用 NVM 作为 DFS 中的持久性存储介质。由于数据存储的延迟已经极大地降低,传统的基于 TCP/IP 网络协议栈的网络通信成为了新的性能瓶颈。新兴的远程内存直接访问(Remote Direct Memory Access,RDMA)技术支持用户态直接访问远端节点的内存,减少了大量的数据复制,很大程度上缓解了这一问题。综合利用 NVM 和 RDMA 技术的已有成果如 Octopus\cite{octopus2017}、Orion\cite{orion2019} 和 Assise\cite{assise2019}。它们从不同方面入手优化文件系统的性能。例如,Octopus 改进了分布式事务协议来降低网络请求的开销;Orion 借鉴了 NOVA,通过一个高效的元数据服务器降低维持一致性的开销;Assise 则实现了完善的缓存机制,通过多级备份实现快速的错误恢复。

由于针对基于 NVM 和 RDMA 技术的 DFS 的研究还处于较为早期的阶段,上述这些研究都只是针对较为单一的方面进行优化。虽然部分工作~\cite{orion2019,assise2019}已经报告了较好的可用性,但它们使用的仍是备份策略,忽视了当前高成本 NVM 带来的节省存储空间的需求。文献调研表明,尚未有已发表的工作研究如何以较低的 NVM 空间开销保证系统的高可用性。

\section{本课题解决的问题}
\label{sec:ch1_thiswork}

本课题的目标是在兼顾系统高可用性的同时,尽可能地降低 NVM 上的空间开销。

具体而言,本课题提出一种适合上述 NVM 特性的新型分布式文件系统。它在充分利用 NVM 高带宽、字节粒度访问等特性的同时,利用纠删码技术的优势,以较低的存储代价实现高可用性。
