% !TeX root = ../main.tex

% 中英文摘要和关键字

\begin{abstract}
  可用性是分布式文件系统的重要属性之一。近年来,持久性内存(NVM)技术出现并高速发展,给这一领域带来了新的机遇。通过合理地利用 NVM 高速、字节粒度访问的特性,系统能够以更低的开销、更高的速度提供高可用的文件存储服务。

  本文提出一种面向 NVM 存储的分布式文件系统。该系统使用纠删码提供可用性保障,其性能不低于使用备份策略的传统分布式文件系统,并且在相同容错条件下显著节省所需的存储空间。测试显示。

  % 关键词用“英文逗号”分隔
  \thusetup{
    keywords = {NVM, RDMA, 分布式文件系统, 可用性},
  }
\end{abstract}

\begin{abstract*}
  Availability is a key feature of distributed file systems. In recent years, the advent and rapid development of non-volatile memory (NVM) technology brings new chances to this field. By leveraging the high-performance and byte-addressable features of NVMs, the system can provide file storage service with lower cost, higher speed, and together with high availability.

  In this paper, we present a new distributed file system for NVM-based storage, which uses erasure codes to guarantee its availability. It performs no worse than traditional systems using replication as their availability strategy, while using far less storage with the same fault-tolerance ability. Evaluations show that. 

  \thusetup{
    keywords* = {NVM, RDMA, distributed file system, availability},
  }
\end{abstract*}
