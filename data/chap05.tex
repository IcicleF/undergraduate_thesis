% !TeX root = ../main.tex

\chapter{结论}
\label{cha:conclusion}

% 概括说明所做工作的情况和价值,指出其中存在的问题和今后的改进方向,对工作中遇到的重要问题要着重指出,并提出自己的见解。

本课题的初衷在于证明,对于使用 NVM 作为持久性存储介质、RDMA 作为网络通信协议栈的分布式文件系统,为了提供数据可用性保障而应用纠删码策略是一个可行的选择。为此,本课题实现了一个基于 NVM+RDMA,并且使用纠删码提供高可用性保障的文件系统原型。接下来,本课题对纠删码编解码和传输延迟、文件系统读写延迟和带宽,以及文件系统可用性水平进行了测试。测试结果表明,使用纠删码虽然引入了额外的计算开销,但相较于传统多副本策略,文件系统的读性能有最多 30\% 的降低,而写性能有最多 11\% 的提升。这表明纠删码策略的性能并不逊色于传统的数据备份策略,揭示了分布式文件系统中应用纠删码作为可用性策略的潜力。

然而,综观本课题的设计、实现与测试,其中仍然存在一些主要问题:

\begin{enumerate}[(1)]
    
    \item 本课题利用 LocoFS 实现了一个文件系统,但 LocoFS 本身的操作延迟较大,不能代表分布式文件系统领域的最新技术水平。由于这与我的主要研究目标关系不大,同时也因为技术和时间有限,本课题并未将文件系统操作逻辑的优化作为研究重点,而是直接使用了现成的 LocoFS 源代码。最近发表的一些工作使用了更加合理的空间分配策略\cite{orion2019},也引入了完善的缓存机制以减小延迟\cite{assise2019};这使它们的写延迟降低到约 $\SI{10}{\us}$,读延迟则降低到 3 $\sim$ $\SI{4}{\us}$。因此,在数据读写延迟与纠删码的编解码延迟处于同一个数量级时,本课题的研究结论在何种程度上仍然有效,尚需要进一步研究;
    
    \item 本课题在测试时使用的服务器集群规模非常小。实际的系统中通常包含大量的节点,例如,一项已有工作在多达 300 $\sim$ 400 个节点构成的缓存系统上应用了 Reed-Solomon 编码\cite{infinicache2020}。在这一规模的系统中,通常还需要引入一致性哈希等机制,方能有效处理空间分配、存储节点增删等情况。因此,基于 NVM+RDMA 的大规模分布式系统中纠删码应当如何应用,尚需要进一步研究;
    
    \item 本课题实现的纠删码策略只有 2 种,且均为原型设计。然而,工程中实际使用的是更为复杂的纠删码编码方案。例如,\ref{subsec:ch1_ec_relworks} 小节中提到了 Facebook 公司的 f4 系统,该系统虽然使用朴素的 Reed-Solomon 编码,但使用了复杂的存储空间分配模式,以应对实际应用中不同规模的故障;同样于该小节中提到的 Windows Azure 存储系统则使用了一个称为 LRC(Local Reconstruction Code)的编码策略,它利用大多数故障都是单节点故障的统计规律,对数据进行二级编码,有效降低了编码和数据恢复的开销。因此,使用更为复杂的纠删码系统时,本课题的研究结论在何种程度上仍然有效,尚需要于进一步研究。

\end{enumerate}

综上所述,本课题的研究成果尚有一些不完整之处,今后应当按照这些方向加以改进。在有一定改进空间的同时,本课题确实地证明了使用纠删码构建高可用分布式持久内存文件系统的可行性和有效性,完成了预设的研究目标。
